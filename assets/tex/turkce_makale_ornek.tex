\documentclass[a4paper,12pt]{article}
\usepackage[utf8]{inputenc}   % Türkçe karakterler için
\usepackage[T1]{fontenc}      % Türkçe karakterler için
\usepackage[turkish, shorthands=:!]{babel}   % Türkçe dil desteği.
\usepackage[top=3cm,
		    bottom=2cm,
		    left=2.5cm,
		    right=2cm]{geometry}%Kenar boşlukları ayarı
\usepackage{newtxtext}	%Font 

\title{\LaTeX{} ile Örnek Türkçe Makale Şablonu}
\author{Yazar Adı}
\date{5.10.2025}

\begin{document}

\maketitle

\begin{abstract}
Bu belge, Türkçe dilinde yazılmış minimal bir \LaTeX{} makale şablonudur.
\end{abstract}

\section{Giriş}
\sloppy % Overfull \hbox (2.75262pt too wide) uyarısını almamak için gerekli komut. Kullanılmayabilir.
Türkçe belge oluştururken, \verb|\usepackage[turkish, shorthands=:!]{babel}| kullanımına dikkat edilmesi gerekir. \verb|"shorthands=:!"| opsiyonu, Türkçe belgelerde özellikle resim eklemelerde \verb|"="| gibi karakterleri kullanırken hata mesajlarını engeller.


\section{Sonuç}
Sonuçlarınızı burada özetleyebilirsiniz.

\end{document}