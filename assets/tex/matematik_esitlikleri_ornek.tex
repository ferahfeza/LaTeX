\documentclass[a4paper,12pt]{article}
\usepackage[utf8]{inputenc}
\usepackage[T1]{fontenc}
\usepackage[turkish]{babel}
\usepackage{amsmath}    % Gelişmiş matematik komutları
\usepackage{amssymb}    % Ek semboller
\usepackage{verbatim}   % verbatim ortamı için
\usepackage{xcolor}     % Renkli semboller için

\title{\LaTeX{} ile Matematik Eşitlikleri Yazımı}
\author{Ali İhsan Çanakoğlu}
\date{\today}

\begin{document}

\maketitle

\section{Giriş}
Bu dokümanda, matematiksel eşitliklerin \LaTeX{} ile nasıl yazıldığını örneklerle göstereceğiz.
Önce kodu \texttt{verbatim} ortamında vereceğiz, ardından \LaTeX{} Çıktısıni göstereceğiz.
Ayrıca, matrislerin yazımı ve sembollerin renklendirilmesi gibi ileri konular da ele alınacaktır.

\section{Satır İçi (Inline) Matematik}
\subsection{Kod}
\begin{verbatim}
Fonksiyonumuz $f(x)=ax^2+bx+c$ olsun.
\end{verbatim}

\subsection{\LaTeX{} Çıktısı}
Fonksiyonumuz $f(x)=ax^2+bx+c$ olsun.

\section{Numaralı Eşitlik Ortamı}
\subsection{Kod}
\begin{verbatim}
\begin{equation}
a^2 + b^2 = c^2
\end{equation}
\end{verbatim}

\subsection{\LaTeX{} Çıktısı}
\begin{equation}
a^2 + b^2 = c^2
\end{equation}

\section{Birden Fazla Satırlı Eşitlik (align)}
\subsection{Kod}
\begin{verbatim}
\begin{align}
x^2 + y^2 &= r^2 \\
x &= r \cos\theta \\
y &= r \sin\theta
\end{align}
\end{verbatim}

\subsection{\LaTeX{} Çıktısı}
\begin{align}
x^2 + y^2 &= r^2 \\
x &= r \cos\theta \\
y &= r \sin\theta
\end{align}

\section{Matris Yazımı}
\subsection{Kod}
\begin{verbatim}
\[
A = \begin{bmatrix}
a_{11} & a_{12} \\
a_{21} & a_{22}
\end{bmatrix}
\]
\end{verbatim}

\subsection{\LaTeX{} Çıktısı}
\[
A = \begin{bmatrix}
a_{11} & a_{12} \\
a_{21} & a_{22}
\end{bmatrix}
\]

\section{Sembollerin Renklendirilmesi}
\subsection{Kod}
\begin{verbatim}
\[
E = m\textcolor{red}{c}^2
\]
\end{verbatim}

\subsection{\LaTeX{} Çıktısı}
\[
E = m\textcolor{red}{c}^2
\]

\subsection{Birden Fazla Renkli Sembol}
\begin{verbatim}
\[
\textcolor{blue}{x}^2 + \textcolor{green}{y}^2 = \textcolor{orange}{r}^2
\]
\end{verbatim}

\subsection{\LaTeX{} Çıktısı}
\[
\textcolor{blue}{x}^2 + \textcolor{green}{y}^2 = \textcolor{orange}{r}^2
\]

\section{Matematiksel Sembol Örnekleri}
\subsection{Kod}
\begin{verbatim}
$\alpha, \beta, \gamma, \int_0^1 x\,dx, \sum_{n=1}^{\infty} \frac{1}{n^2}$
\end{verbatim}

\section{İntegral, Toplam ve Çarpım Örnekleri}

Aşağıda çeşitli integral, toplam ve çarpım örneklerinin önce kodları (\texttt{verbatim} ortamında), ardından renderlanmış halleri sunulmuştur.

\subsection{Temel İntegral}
\subsubsection{Kod}
\begin{verbatim}
\[
\int_a^b f(x)\,dx
\]
\end{verbatim}

\subsubsection{\LaTeX{} Çıktısı}
\[
\int_a^b f(x)\,dx
\]

\subsection{Belirli İntegral}
\subsubsection{Kod}
\begin{verbatim}
\[
\int_0^1 x^2\,dx
\]
\end{verbatim}

\subsubsection{\LaTeX{} Çıktısı}
\[
\int_0^1 x^2\,dx
\]

\subsection{Çift İntegral}
\subsubsection{Kod}
\begin{verbatim}
\[
\iint\limits_{D} xy\,dx\,dy
\]
\end{verbatim}

\subsubsection{\LaTeX{} Çıktısı}
\[
\iint\limits_{D} xy\,dx\,dy
\]

\subsection{Toplam (Sum) Örneği}
\subsubsection{Kod}
\begin{verbatim}
\[
\sum_{n=1}^{10} n^2
\]
\end{verbatim}

\subsubsection{\LaTeX{} Çıktısı}
\[
\sum_{n=1}^{10} n^2
\]

\subsection{Sonsuz Toplam}
\subsubsection{Kod}
\begin{verbatim}
\[
\sum_{k=0}^{\infty} \frac{1}{2^k}
\]
\end{verbatim}

\subsubsection{\LaTeX{} Çıktısı}
\[
\sum_{k=0}^{\infty} \frac{1}{2^k}
\]

\subsection{Çarpım (Product) Örneği}
\subsubsection{Kod}
\begin{verbatim}
\[
\prod_{i=1}^n i
\]
\end{verbatim}

\subsubsection{\LaTeX{} Çıktısı}
\[
\prod_{i=1}^n i
\]

\subsection{Renkli Sembol ile Toplam}
\subsubsection{Kod}
\begin{verbatim}
\[
\sum_{n=1}^{\textcolor{red}{N}} \textcolor{blue}{a}_n
\]
\end{verbatim}

\subsubsection{\LaTeX{} Çıktısı}
\[
\sum_{n=1}^{\textcolor{red}{N}} \textcolor{blue}{a}_n
\]

\subsection{Renkli Sembol ile İntegral}
\subsubsection{Kod}
\begin{verbatim}
\[
\int_0^{\textcolor{green}{T}} \textcolor{magenta}{f}(t)\,dt
\]
\end{verbatim}

\subsubsection{\LaTeX{} Çıktısı}
\[
\int_0^{\textcolor{green}{T}} \textcolor{magenta}{f}(t)\,dt
\]

\section{Matematiksel Sembol Örnekleri}
\subsection{Kod}
\begin{verbatim}
$\alpha, \beta, \gamma, \int_0^1 x\,dx, \sum_{n=1}^{\infty} \frac{1}{n^2}$
\end{verbatim}

\subsection{\LaTeX{} Çıktısı}
$\alpha, \beta, \gamma, \int_0^1 x\,dx, \sum_{n=1}^{\infty} \frac{1}{n^2}$


\end{document}