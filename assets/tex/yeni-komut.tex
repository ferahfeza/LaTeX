\documentclass{article}
% Örnek paketler
\usepackage[T1]{fontenc}
\usepackage{lmodern}
\usepackage{amsmath,amssymb}

% 1) Basit komut (argümansız)
\newcommand{\R}{\mathbb{R}}

% 2) Tek zorunlu argümanlı komut
\newcommand{\boldit}[1]{\textbf{\textit{#1}}}

% 3) Bir opsiyonel ve bir zorunlu argümanlı komut
%    Kullanımı: \todo{iş} veya \todo[ÖNEMLİ]{iş}
\newcommand{\todo}[2][Not:]{\textbf{[#1] #2}}

% 4) Eğer var olan bir komutu değiştirmek isterseniz:
% \renewcommand{\today}{\number\day/\number\month/\number\year}

\begin{document}

Bu bir örnek dokümandır.

Matematikte sıklıkla kullandığımız sayı kümesi: \(\R\).

Metin vurgulama örneği: \boldit{Bu metin kalın ve italik.}

Opsiyonel argümanlı not komutu:
\todo{Bunu daha sonra düzenle.}

Opsiyonel argüman ile:
\todo[ACİL]{Sunum için slayt hazırla.}

\end{document}