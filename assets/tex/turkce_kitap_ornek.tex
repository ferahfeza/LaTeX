\documentclass[a4paper,12pt,openany]{book}
\usepackage[utf8]{inputenc}   % Türkçe karakterler için
\usepackage[T1]{fontenc}      % Türkçe karakterler için
\usepackage[turkish, shorthands=:!]{babel}   % Türkçe dil desteği.   % Türkçe dil desteği
\usepackage[top=3cm,
		    bottom=2cm,
		    left=2.5cm,
		    right=2cm]{geometry}%Kenar boşlukları ayarı
\usepackage{newtxtext}	%Font
\usepackage{xcolor} % Renk tanımlama için
\usepackage[
    colorlinks=true,
    linkcolor=teal,          % İç bağlantı rengi
    citecolor=blue,         % Atıf bağlantı rengi
    urlcolor=magenta        % URL bağlantı rengi
]{hyperref}
\usepackage{graphicx}

\title{Örnek Türkçe Kitap}
\author{Yazar Adı}
\date{5.10.2025}

\begin{document}

\maketitle

\frontmatter
\tableofcontents

\mainmatter

\chapter{Giriş}
Bu bölümde kitabınızın konusuna giriş yapabilirsiniz.

\section{Referans Gösterme}
\LaTeX{}'in geliştiricisi olan Leslie Lamport  kitabında sistemi tanıtmıştır \cite{latexkitap}.

\section{Resim Ekleme}
Dokümana resim ekleyebilmek için \verb|\usepackage{graphicx}| paketi gereklidir. Resmi başlıksız kullanacaksanız,
\begin{verbatim}
\includegraphics[scale=1]{'resim_dosyası_adı'}
\end{verbatim} 
komutu yeterlidir. Resim başlığı ve numaralandırma gibi özellikler için \verb|\begin{figure}\end{figure}| tagı içine yerleştirmelisiniz.
\begin{verbatim}
\begin{figure}[htbp!]
\includegraphics[scale=1]{'resim_dosyası_adı'}
\caption{Resim alt yazısı.}\label{•}
\end{figure}
\end{verbatim}
\begin{figure}[htbp!]
\centering % Sayfaya ortalama.
\includegraphics[scale=0.5]{example-image-a}
\caption{Örnek Resim.}\label{resim:1}
\end{figure}

Resme atıf yapmak için, resim tanımlamasında \verb|\label{•}| tanımını yaptık. Metin içinde \verb|\ref{resim:1}| komutu ile link ataması yapılır.
Şekil \ref{resim:1}'de örnek bir resim görülmektedir.

\chapter{Teori}
Burada teorik bilgileri sunabilirsiniz.


\begin{thebibliography}{9}

\bibitem{latexkitap}
Leslie Lamport,
\textit{LaTeX: A Document Preparation System},
Addison-Wesley, 2. Baskı, 1994.

\end{thebibliography}

\end{document}