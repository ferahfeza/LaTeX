\documentclass[conference,a4paper]{IEEEtran}

% Unicode + Türkçe desteği (XeLaTeX veya LuaLaTeX ile derleyin; pdflatex ile de çalışır)
\usepackage{fontspec}           % Xe/LuaTeX için (pdflatex'de yoksa hata verir; pdflatex kullanacaksanız bu satırı kaldırın ve alt satırdaki pdflatex önerisini kullanın)
\usepackage{polyglossia}
\setdefaultlanguage{turkish}

\setmainfont{TeX Gyre Termes}
\setsansfont{TeX Gyre Heros}
\setmonofont{TeX Gyre Cursor}

\usepackage{amsmath,amssymb}
\usepackage{siunitx}
\usepackage{hyperref}
\usepackage{cite}
\usepackage{microtype}

\title{Coulomb Kanunu: Tarihçe, Temassız Kuvvet Özelliği ve Foton Teorisi ile İlişkisi}

\author{\IEEEauthorblockN{Ferah Feza}
\IEEEauthorblockA{Independent Researcher\\
Email: \texttt{ferahfeza@example.com}}
}

\begin{document}
\maketitle

\begin{abstract}
Bu çalışma Coulomb kanununu matematiksel ifadeleriyle birlikte açıklar, kanunun tarihçesine kısaca değinir, temassız (action-at-a-distance) kuvvet kavramı ve alan yaklaşımı çerçevesinde kanunun fiziksel yorumunu sunar ve son olarak klasik elektromanyetizma ile Kuantum Elektrodinamik (QED) bağlamında Coulomb etkileşiminin foton teorisiyle ilişkisini tartışır.
\end{abstract}

\begin{IEEEkeywords}
Coulomb kanunu, elektrostatik, temassız kuvvet, foton, kuantum elektrodinamik
\end{IEEEkeywords}

\section{G\.{I}R\.{I}\c{S}}
Elektrik yükleri arasındaki etkileşimlerin en temel ifadesi Coulomb kanunudur. İki nokunsal yük arasındaki elektrostatik kuvvetin büyüklüğünü ve yönünü verir. Coulomb kanunu hem klasik elektrodinamik hesaplamaların temelini oluşturur hem de kuantum alan kuramlarında etkileşimin nasıl ara öğeler (sanal parçacıklar) aracılığıyla aktarıldığını anlamada köprü görevi görür.

\section{COULOMB KANUNUNUN MATEMAT\.{I}KSEL \.{I}FADESI}
İki nokunsal yük \(q_1\) ve \(q_2\) arasındaki Coulomb kuvvetinin vektörel ifadesi:
\begin{equation}
\mathbf{F}_{12} = k_\mathrm{e}\,\frac{q_1 q_2}{r^2}\,\hat{\mathbf{r}}_{12},
\end{equation}
burada \(r\) iki yük arasındaki uzaklık, \(\hat{\mathbf{r}}_{12}\) \(q_1\) noktasından \(q_2\) noktasına doğru birim vektördür ve
\begin{equation}
k_\mathrm{e}=\frac{1}{4\pi\varepsilon_0}
\end{equation}
Coulomb sabitidir. SI birimlerinde \(\varepsilon_0\) boşluğun elektriksel geçirgenliği olup yaklaşık \(\varepsilon_0 \approx 8.854\,187\,817\times10^{-12}\,\mathrm{F\cdot m^{-1}}\)'dir.

Elektrostatik potansiyel:
\begin{equation}
V(\mathbf r) = \frac{1}{4\pi\varepsilon_0}\frac{q}{r},
\end{equation}
ve elektrik alan \(\mathbf{E}=-\nabla V\) ile verilir. Birden fazla yük için alanların süperpozisyon ilkesi geçerlidir.

\section{TAR\.{I}H\c{C}E VE DENEYSEL KANITLAR}
Coulomb kanunu adı Charles-Augustin de Coulomb'a (1736--1806) atfedilir. Coulomb, 1785'te torsiyon terazisiyle yaptığı hassas deneylerle elektrik yükleri arasındaki kuvvetin uzaklığın karesi ile ters orantılı olduğunu gösterdi. Maxwell ve sonraki çalışmalarda bu gözlemler daha genel elektromanyetik kuramın (Maxwell denklemleri) parçası hâline getirildi. Modern deneyler Coulomb davranışını geniş aralıklarda doğrulamış, fotonun kütlesine ilişkin çok küçük üst sınırlar koymuştur.

\section{TEMASSIZ KUVVET \.{O}ZELL\.{I}Ğ\.{I} VE ALAN KAVRAMI}
Coulomb kuvveti, iki yük doğrudan temas hâlinde olmasa bile birbirini etkiler; bu nedenle tarihsel olarak ``temassız'' (action-at-a-distance) etkileşim olarak anılmıştır. Ancak 19. yüzyıldan itibaren alan kavramı etkileşimin daha sezgisel bir açıklamasını sundu:
\begin{itemize}
  \item Yükler etraflarında bir elektrik alanı \(\mathbf{E}(\mathbf r)\) oluşturur.
  \item Başka bir yük bu alanda \(\mathbf{F}=q\mathbf{E}\) kuvvetini hisseder.
  \item Gauss yasası
  \begin{equation}
  \nabla\cdot\mathbf{E}=\frac{\rho}{\varepsilon_0}
  \end{equation}
  alan yaklaşımının matematiksel temelidir.
\end{itemize}

\subsection{ANINDA ETK\.{I} YANILGISI VE GEC\.{I}KME}
Sabit (statik) yük dağılımları için Coulomb yasası anlık bir ilişki gibi görünse de Maxwell denklemlerinin tam relativistik çözümlerinde değişiklikler ışık hızında yayılır; zamanla değişen durumlarda potansiyeller gecikmeli (retarded) çözümlere sahiptir. Ayrıca gauge seçimleri (ör. Coulomb gauge) bazı bileşenleri anlıkmış gibi gösterse de fiziksel sinyaller nedenselliğe uygun olarak ışık hızını aşmaz.

\section{FOTON TEOR\.{I}S\.{I} İLE İLİ\c{S}KİSİ (KUANTUM ELEKTRODİNAMİK)}
Klasik elektromanyetizma alan kavramıyla etkileşimi açıklar; Kuantum Elektrodinamiği (QED) bu etkileşimi kuantum alan kuramı bağlamında yeniden ifade eder. QED'de elektromanyetik etkileşimler foton alanının kuantumları (fotonlar) aracılığıyla aktarılır.

Önemli noktalar:
\begin{itemize}
  \item Statik Coulomb potansiyeli, QED'de \textit{sanal} foton değiş tokuşunun etkisi olarak yorumlanır. Klasik \(1/r\) potansiyeli, foton propagatörünün uygun limitinin Fourier dönüşümüyle elde edilir.
  \item Sanal fotonlar, gerçek (on-shell) fotonlardan farklı olarak ara taşıyıcıdır; enerji-momentum ilişkisini zorunlu kılmazlar ve doğrudan gözlemlenemezler.
  \item Fotonun kütlesiz olması Coulomb potansiyelinin uzun menzilli \(1/r\) davranışını sağlar. Eğer foton küçük de olsa bir kütleye sahip olsaydı, potansiyel Yukawa tipi olurdu:
  \begin{equation}
  V(r) \propto \frac{e^{-m_\gamma c r/\hbar}}{r},
  \end{equation}
  bu durumda etkileşim kısa menzilli hale gelirdi. Deneyler fotonun kütlesinin çok küçük olduğunu göstermektedir.
  \item QED, vakum kutuplaşması ve renormalizasyon gibi kuantum düzeltmeleriyle klasik potansiyelde küçük değişiklikler öngörür; bunlar atom spektroskopisinde (ör. Lamb kayması) ölçülebilir.
\end{itemize}

\section{S\.{I}NLAR, D\.{O}ZELT\.{I}RMELER VE MODERN DENEYLER}
Coulomb yasası nokunsal yük idealizasyonuna dayanır; gerçekte parçacıklar yapısal ve kuantum etkilerine sahiptir. Çok küçük ölçeklerde QED etkileri, vakum kutuplaşması ve çekirdek içi yapılar önemli olur. Fotonun kütlesi, yeni kuvvet taşıyıcıları veya yeni fizik arayışları Coulomb davranışını hassas deneylerle test etme sebebidir.

\section{SONUÇ}
Coulomb kanunu, basit matematiksel formu ve geniş deneysel doğrulanmışlığı sayesinde hem klasik hem de kuantum düzeyde etkileşimlerin anlaşılmasında merkezi bir rol oynar. Temassız kuvvet kavramı alan yaklaşımı ile daha sağlam şekilde formüle edilmiştir; kuantum alan kuramında ise foton değiş tokuşu ile yeniden yorumlanır. Klasik ve kuantum teoriler arasında güçlü bir tutarlılık mevcuttur; kuantum düzeltmeler yalnızca küçük farklılıklar getirir.

\section*{TEŞEKKÜR}
Bu doküman önceki çalışmalardan derlenmiş bilgileri Türkçe olarak IEEE formatında sunmak amacıyla yeniden düzenlenmiştir.

\begin{thebibliography}{9}
\bibitem{coulomb1785} C.-A. Coulomb, ``Histoire de l'Académie Royale des Sciences,'' 1785.
\bibitem{maxwell} J. C. Maxwell, \textit{A Treatise on Electricity and Magnetism}, 1873.
\bibitem{jackson} J. D. Jackson, \textit{Classical Electrodynamics}, 3rd ed., Wiley, 1998.
\bibitem{feynman} R. P. Feynman, R. B. Leighton, M. Sands, \textit{The Feynman Lectures on Physics}, Vol. II.
\bibitem{peskin} M. E. Peskin, D. V. Schroeder, \textit{An Introduction to Quantum Field Theory}, Addison-Wesley, 1995.
\end{thebibliography}

\end{document}